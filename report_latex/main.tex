\documentclass{article}
\usepackage{graphicx}
\usepackage{float}
\usepackage{hyperref}
\usepackage{listings}
\usepackage{color}
\usepackage{geometry}
\geometry{a4paper, margin=1in}

\title{Distributed Systems Project: Mesh Topology Communication Analysis}
\author{Aniket Gupta}
\date{\today}

\begin{document}

\maketitle

\section{Introduction}
This project explores communication algorithms on 2D and 3D mesh topologies using MPI. We implemented and analyzed Broadcast and Gather operations, comparing Dimension-Order Routing (DOR) with a Flooding algorithm.

\section{Mesh Topology Implementation}
We implemented a generalized mesh topology that adapts to the number of available processes ($P$).
\begin{itemize}
    \item \textbf{2D Mesh:} $N \times M$ grid where $N \times M \approx P$.
    \item \textbf{3D Mesh:} $N \times M \times K$ grid where $N \times M \times K \approx P$.
\end{itemize}
The topology is constructed using MPI communicators and Cartesian mapping.

\section{Algorithms}

\subsection{Dimension-Order Routing (DOR)}
Standard broadcast/gather propagates along axes sequentially (e.g., X then Y then Z).
\begin{itemize}
    \item \textbf{Broadcast:} Root sends to row leaders, who then send to their columns.
    \item \textbf{Gather:} Leaves send to row leaders, who aggregate and send to root.
\end{itemize}

\subsection{Flooding (Wavefront)}
A flooding algorithm was implemented to simulate wavefront propagation.
\begin{itemize}
    \item Nodes receive data from any neighbor at distance $d$.
    \item Nodes forward data to all neighbors at distance $d+1$.
    \item This approach minimizes latency in irregular meshes but increases message complexity.
\end{itemize}

\section{Performance Analysis}

\subsection{Experimental Setup}
\begin{itemize}
    \item \textbf{Environment:} Local 16-core machine using OpenMPI.
    \item \textbf{Process Counts:} 4, 8, 9, 16.
    \item \textbf{Data Sizes:} 100 to 10,000 elements.
\end{itemize}

\subsection{Results}

\begin{figure}[H]
    \centering
    \includegraphics[width=1.0\textwidth]{../results/real_performance_vs_datasize.png}
    \caption{Performance vs Data Size for 2D/3D Broadcast and Gather, including Flooding.}
    \label{fig:perf_vs_size}
\end{figure}

\begin{figure}[H]
    \centering
    \includegraphics[width=1.0\textwidth]{../results/real_2d_vs_3d_comparison.png}
    \caption{Comparison of 2D vs 3D Mesh Performance.}
    \label{fig:2d_vs_3d}
\end{figure}

\begin{figure}[H]
    \centering
    \includegraphics[width=1.0\textwidth]{../results/speedup_analysis.png}
    \caption{Speedup Analysis of 3D over 2D Mesh.}
    \label{fig:speedup}
\end{figure}

\section{Observations}
\begin{enumerate}
    \item \textbf{Flooding Performance:} The flooding algorithm demonstrated significantly lower latency for large data sizes compared to standard DOR, likely due to better link utilization and pipelining effects in the simulation.
    \item \textbf{3D vs 2D:} 3D meshes generally outperformed 2D meshes for larger process counts due to lower network diameter.
    \item \textbf{Scalability:} The algorithms scaled as expected with data size, following the linear latency-bandwidth model.
\end{enumerate}

\section{Conclusion}
The project successfully demonstrated the trade-offs between different topology dimensions and routing algorithms. The flooding algorithm proved to be a highly efficient alternative for broadcast operations in this setup.

\end{document}
